\section{Pendahuluan}
\subsection{Latar Belakang}
Perkembangan teknologi informasi dan komunikasi saat ini sangat pesat, terutama dalam hal konektivitas yang semakin mengandalkan jaringan nirkabel (wireless). Baik di lingkungan rumah, kantor, sekolah, maupun tempat umum, jaringan wireless seperti Wi-Fi telah menjadi infrastruktur penting dalam menunjang aktivitas digital sehari-hari. Kebutuhan akan mobilitas, fleksibilitas, dan kemudahan akses informasi membuat jaringan wireless menjadi pilihan utama dibanding jaringan kabel (wired) yang cenderung terbatas oleh instalasi fisik. \\ Namun, kemudahan yang ditawarkan jaringan wireless juga membawa tantangan tersendiri, seperti kestabilan koneksi, keamanan jaringan, dan pemilihan perangkat yang sesuai. Permasalahan seperti sinyal lemah di area tertentu, koneksi tidak aman, atau perangkat tidak kompatibel sering kali terjadi dan membutuhkan pemahaman teknis yang baik agar dapat diatasi secara efektif. \\ Praktikum jaringan wireless ini dilaksanakan sebagai bentuk penguatan pemahaman konsep dan implementasi teknologi wireless di dunia nyata. Melalui praktikum ini, mahasiswa tidak hanya akan memahami teori dasar seperti perbedaan wired dan wireless, fungsi access point, router, dan wireless NIC, tetapi juga akan mampu merancang solusi konektivitas wireless sesuai dengan kebutuhan spesifik suatu lokasi atau skenario tertentu. \\ Topik ini menjadi sangat penting dan relevan di era Internet of Things (IoT), smart home, cloud computing, dan mobile computing yang semuanya mengandalkan jaringan wireless sebagai tulang punggung komunikasi data. Oleh karena itu, pemahaman yang mendalam dan keterampilan praktis dalam mengelola jaringan wireless merupakan bekal penting bagi calon profesional di bidang teknologi informasi, jaringan komputer, dan telekomunikasi.

\subsection{Dasar Teori}
Dalam memahami dan mengimplementasikan jaringan wireless secara efektif, diperlukan penguasaan teori-teori dasar terkait konsep komunikasi data nirkabel, perangkat jaringan, serta standar dan protokol yang digunakan. Bagian ini menjelaskan prinsip-prinsip dasar yang menjadi landasan pelaksanaan praktikum. \\ Jaringan wireless adalah jaringan komunikasi yang menggunakan gelombang elektromagnetik, seperti gelombang radio atau inframerah, untuk menghubungkan antar perangkat tanpa media kabel fisik. Media transmisi ini memungkinkan fleksibilitas, mobilitas tinggi, dan kemudahan dalam instalasi, menjadikannya pilihan populer dalam berbagai lingkungan, baik rumah, kantor, maupun ruang publik. \\ Tipe jaringan wireless yaitu Wi-Fi merupakan Teknologi berbasis standar IEEE 802.11 yang digunakan untuk menghubungkan perangkat ke jaringan lokal dan internet tanpa kabel. Wi-Fi memungkinkan kecepatan tinggi dan jangkauan cukup luas. Dan Bluetooth merupakan Protokol komunikasi jarak dekat (short-range) yang digunakan untuk koneksi antar perangkat secara langsung (peer-to-peer), seperti menghubungkan ponsel ke speaker atau headset. \\ Standar ini mendefinisikan arsitektur dan mekanisme komunikasi pada jaringan Wireless LAN (WLAN). Versi yang terkenal meliputi : 802.11b/g/n/ac/ax Menentukan frekuensi operasi (2.4 GHz / 5 GHz), kecepatan maksimum, dan fitur seperti MIMO (Multiple Input Multiple Output). Dan SSID (Service Set Identifier): Nama jaringan wireless yang membedakan satu WLAN dari yang lain. \\ Karena media transmisinya terbuka di udara, jaringan wireless sangat rentan terhadap penyadapan dan akses tidak sah. Untuk mengatasi hal ini digunakan berbagai protokol keamanan: WEP (Wired Equivalent Privacy): Protokol lama, kini tidak direkomendasikan karena mudah dibobol, WPA / WPA2 / WPA3: Protokol keamanan modern yang menggunakan enkripsi lebih kuat seperti TKIP dan AES untuk melindungi integritas dan kerahasiaan data. \\ Dengan memahami teori-teori di atas, peserta praktikum diharapkan mampu menerapkan konsep dan prinsip ilmiah yang relevan untuk menganalisis, merancang, dan mengevaluasi jaringan wireless secara sistematis dan tepat guna.

%===========================================================%
\section{Tugas Pendahuluan}
\begin{enumerate}
	\item Tidak ada yang secara mutlak lebih baik antara jaringan wired dan wireless—keduanya memiliki keunggulan dan kekurangan masing-masing tergantung kebutuhan pengguna. \\ - Jaringan wired (berkabel) unggul dalam hal kestabilan, kecepatan tinggi, dan keamanan fisik. Sangat cocok untuk perangkat tetap seperti PC, server, dan perangkat di studio editing atau data center. \\ - Jaringan wireless (nirkabel) unggul dari sisi mobilitas, kemudahan instalasi, dan fleksibilitas. Sangat sesuai untuk laptop, smartphone, dan perangkat yang sering berpindah tempat.
	\item Perbedaan router, access point, dan modem : \\ - Router : Menghubungkan jaringan lokal (LAN) ke jaringan lain seperti internet. \\ - Access point : Menghubungkan perangkat wireless (HP, laptop) ke jaringan kabel. \\ - Modem : Mengubah sinyal digital dari perangkat menjadi sinyal analog untuk transmisi melalui saluran ISP (misal kabel/ADSL), dan sebaliknya.
	\item Perangkat Wireless Bridge, karena dirancang khusus untuk menghubungkan dua titik jaringan yang berjauhan tanpa menggunakan kabel, mampu menjangkau jarak lebih dari 10 km dengan sinyal terarah dan stabil, Dengan menggunakan dua unit Wireless Bridge (point-to-point), jaringan dari gedung A dapat diteruskan secara nirkabel ke gedung B, lalu didistribusikan ke perangkat melalui switch atau AP lokal.
\end{enumerate}