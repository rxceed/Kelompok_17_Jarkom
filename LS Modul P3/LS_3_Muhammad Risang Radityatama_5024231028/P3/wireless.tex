\section{Pendahuluan}
\subsection{Latar Belakang}
Pada masa-masa awal berkembangnya internet, perangkat-perangkat end device atau end host yang dapat mengakses internet adalah komputer yang terhubung dengan sebuah router menggunakan kabel LAN, kemudian router tersebut terhubung dengan router lain dengan menggunakan kabel pula. Jaringan internet yang terhubung menggunakan kabel ini tentu saja merupakan solusi yang cukup dan tepat untuk menjawab masalah komunikasi antar perangkat. Namun, seiring dengan berkemabnganya zaman, kebutuhan akan akses internet semakin meningkat dan di antara kebutuhan-kebutuhan tersebut terdapat perangkat-perangkat yang membutuhkan akses internet atau ke dalam sebuah jaringan namun perangkat tersebut tidak bisa terhubung ke dalam jaringan secara fisik melalui kabel. Untuk mengatasi masalah ini, maka diciptakanlah teknologi jaringan nirkabel (wireless). Jaringan nirkabel memungkinkan perangkat host untuk terhubung ke dalam jaringan tanpa harus terhubung secara fisik melalui kabel dengan menggunakan sinyal elektromagnetik sebagai perantara antara perangkat dengan jaringan. Salah dua jenis jaringan nirkabel yang terkenal adalah Wi-Fi dan Bluetooth. Kemampuan jaringan nirkabel untuk menghubungkan perangkat dengan jarinagn tanpa harus terhubung dengan kabel membuat keberadaan jaringan nirkabel menjadi sangat penting, karena dengan hilangnya batasan fisik maka perangkat-perangkat mobile seperti handphone dan laptop serta perangkat-perangkat IoT yang tidak memungkinkan untuk terhubung ke router dengan menggunakan kabel dapat terhubung ke internet atau jaringan. Karena pentingnya peran jaringan nirkabel dalam sistem jaringan modern, maka diperlukan pemahaman yang mendalam tentang jaringan nirkabel. Oleh karena itu, untuk memperdalam pemahaman, maka dilaksanakan praktikum jaringan nirkabel.

\subsection{Dasar Teori}
Jaringan nirkabel atau jaringan wireless adalah jaringan yang tidak menggunakan media fisik seperti kabel untuk menghubungkan perangkat-perangkat dalam jaringan, namun sebagai gantinya menggunakan gelombang elegtromagnetik seperti gelombang radio sebagai media penghubung antar perangkat. Karena tidak menggunakan media fisik, jaringan nirkabel menjadi lebih fleksibel dibandingkan jaringan berkabel. layaknya jaringan berkabel konvensional, jaringan nirkabel juga dibagi menjadi beberapa jenis, yaitu WLAN, WMAN, dan WWAN. WLAN atau wireless LAN adalah jaringan lokal yang terdiri atas dua perangkat atau lebih yang terhubung secara nirkabel. Pada umumnya WLAN menggunakan standar IEEE 802.11, atau seringkali disebut juga sebagai Wi-Fi (wireless fidelity). Standar IEEE 802.11 beroperasi pada band frekuensi 2,4 GHz (802.11 ,802.11b, dan 802.11g) atau 5GHz (802.11a), namun ada juga versi yang bekerja pada keduanya (dual band) yaitu 802.11/n. WMAN atau wireless MAN adalah sekumpulan jaringan lokal (LAN) yang terhubung secara nirkabel pada suatu lingkup yang besar seperti wilayah metropolitan/kota, dan WWAN atau wireless WAN adalah sekumpulan LAN, MAN, dan jaringan-jaringan pribadi yang terhubung secara nirkabel dan mencakup wilayah yang sangat besar. Selain ketiga jenis tersebut, ada pula jenis jaringan nirkabel lain seperti Bluetooth. Berbeda dengan Wi-Fi yang dapat digunakan dalam skala masif, Bluetooth diperuntukan untuk penggunaan P2P (peer to peer) dan jarak dekat. Bluetooth beroperasi pada frekuensi 2,4 GHz dan menghubungkan perangkat secara peer to peer. Kelebihan dari jaringan nirkabel dibandingkan jaringan berkabel adalah fleksibilitas (karena tidak perlu terhubung secaara fisik), kemudahan (karena tidak perlu menghubungkan perangkat dengan kabel), dan skalabilitas yang tinggi (karena koneksi tidak terbatas oleh kapasitas kabel).

Dalam membangun sebuah jaringan, baik jaringan berkabel maupun nirkabel, diperlukan perangkat-perangkat jaringan (network device) untuk menghubungkan end device (seperti komputer, laptop, dan handphone) ke dalam jaringan. Untuk membangun sebuah jaringan nirkabel, diperlukan beberapa perangkat-perangkat jaringan yang menjadi inti dari jaringan tersebut. Perangkat-perangkat tersebut adalah access point dan wireless router. Access point (AP) merupakan perangkat yang berfungsi untuk menghubungkan jaringan nirkabel dengan jaringan berkabel. AP membuat sebuah jaringan Wi-Fi yang memungkinkan perangkat-perangkat yang terhubung secara wireless berkomnuikasi dengan perangkat lain di dalam jaringan, dan umumnya digunakan sebagai ekstensi sebuah jaringan atau untuk menyediakan akses Wi-Fi di tempat yang tidak memiliki akses pada jaringan. Wireless router merupakan perangkat yang meneruskat paket-paket data dari perangkat pengirim ke alamat tujuan secara nirkabel melalui jaringan Wi-Fi. Secara fungsi, wireless router sama saja dengan router berkabel, yang membedakan hanya media transmisi datanya. Selain AP dan wireless router, ada juga perangkat jaringan lain dalam jaringan nirkabel yang dapat menunjang kinerja jaringan, contohnya adalah range extender dan wireless bridge. Range extender adalah perangkat yang dapat menerima sinyal Wi-Fi dari jaringan nirkabel lalu kemudian memancarkan ulang sinyal dari jaringan tersebut, sehingga memperluas jangkauan jaringan tersebut. Wireless bridge adalah perangkat yang berfungsi untuk menghubungkan dua jaringan secara nirkabel dengan jarak yang jauh, sehingga berfungsi layaknya sebuah kabel ethernet jarak jauh bagi jaringan nirkabel.

%===========================================================%
\section{Tugas Pendahuluan}
\begin{enumerate}
	\item Jelasin apa yang lebih baik, jaringan wired atau jaringan wireless?\\
	Untuk menilai manakah yang lebih baik anatara jaringan wired atau wireless harus melihat konteks dan kasus penggunannya terlebih dahulu. Untuk jaringan yang diakses oleh berbagai macam perangkat secara dinamis (perangkat yang terhubung ke jaringan berubah-ubah dan banyak jenisnya) dan untuk akses internet secara umum, jaringan wireless lebih baik karena akses ke jaringan wireless hanya perlu menghubungkan perangkat dengan SSID jaringan tanpa harus susah payah mengatur kabel yang di mana tidak semua perangkat yang terhubung ke jaringan bisa terhubung ke jaringan dengan kabel (contohnya handphone), mayoritas perangkat modern juga telah mendukung konektivitas secara wireless sehingga jaringan wireless bisa mencakup lebih banyak perangkat dibanding jaringan wired. Namun, bila yang dibutuhkan adalah jaringan yang aman dan memiliki konektivitas yang stabil, maka jaringan wired lebih baik karena jaringan wired hanya bisa diakses oleh perangkat yang terhubung secara fisik dengan jaringan melaui kabel sehingga meminimalisir risiko peretasan atau kebocoran data, jaringan wired juga memiliki kecepatan transfer lebih stabil dan tinggi tidak dipengaruhi oleh gangguan-gangguan yang dapat dialami sinyal Wi-Fi seperti interferensi atau sinyal yang terhalang tembok.
	\item Apa perbedaan antara router, access point, dan modem?\\
	Perbedaan di antara ketiganya adalah fungsinya dan bagaimana ketiganya menghubungkan perangkat ke internet. Router berfungsi untuk menghubungkan perangkat-perangkat yang terhubung dengannya ke dalam sebuah jaringan LAN dan mengatur lalu lintas komunikasi data dalam jaringan, router juga berfungsi mengubungkan jaringan LAN ke internet. Access point berfungsi untuk menghubungkan jaringan nirkabel ke jaringan berkabel, jadi access point hanya dapat menghubungkan perangkat-perangkat yang terhubung secara nirkabel ke dalam jaringan LAN berkabel yang sudah ada karena access point tidak bisa membuat jaringan LAN sendiri, sehingga untuk dapat mengakses internet dari Wi-Fi yang dipancarakan oleh access point maka access point tersebut harus terhubung ke LAN yang telah terhubung ke internet. Modem berfungsi untuk mengubah sinyal digital menjadi sinyal analog lalu mengirimkannya ke modem penerima untuk kemudian diolah lagi dari sinyal analog menjadi sinyal digital, di mana pengiriman sinyal ini dapat melalui berbagai media, contohnya kabel telepon, kabel TV, dan Wi-Fi. Modem hanya berperan untuk menerima dan meneruskan sinyal dari jaringan yang sudah ada, bukan membuat atau membangun sebuah jaringan baru, sehingga untuk mengakses internet maka sinyal yang dikirim dan diterima oleh modem haruslah sinyal yang berasal dari sebuah jaringan milik internet service provicer (ISP) atau jaringan yang sudah terhubung dan memiliki akses ke internet.
	\item Jika kamu diminta menghubungkan dua ruangan di gedung berbeda tanpa menggunakan kabel, perangkat apa yang kamu pilih? Jelaskan alasannya.\\
	Untuk kasus tersebut, perangkat yang paling cocok digunakan adalah wireless bridge. Wireless bridge cocok digunakan karena wireless bridge dapat menghubungkan dua jaringan point to point secara nirkabel dan memiliki jangkauan jarak yang panjang sehingga cocok untuk menghubungkan jaringan antar gedung. Wirekess bridge juga dapat menghubungkan dua LAN secara nirkabel, asalkan kedua LAN tersebut memiliki protokol yang sama, sehingga wireless bridge berfungsi layaknya sebuah kabel LAN wireless. Dengan begitu, dua ruangan di gedung yang berbeda dapat terhubung dalam satu jaringan LAN tanpa harus menggunakan kabel satu centimeter pun.
\end{enumerate}